% !TEX TS-program = xelatex
% !TEX encoding = UTF-8 Unicode
% !Mode:: "TeX:UTF-8"

\documentclass{resume}
\usepackage{zh_CN-Adobefonts_external} % Simplified Chinese Support using external fonts (./fonts/zh_CN-Adobe/)
%\usepackage{zh_CN-Adobefonts_internal} % Simplified Chinese Support using system fonts
\usepackage{linespacing_fix} % disable extra space before next section
\usepackage{cite}

% Simpler bibsection for CV sections
% (thanks to natbib for inspiration)
\makeatletter
\newlength{\bibhang}
\setlength{\bibhang}{1em} %1em}
\newlength{\bibsep}
 {\@listi \global\bibsep\itemsep \global\advance\bibsep by\parsep}
\newenvironment{bibsection}%
        {\begin{enumerate}{}{%
%        {\begin{list}{}{%
       \setlength{\leftmargin}{\bibhang}%
       \setlength{\itemindent}{-\leftmargin}%
       \setlength{\itemsep}{\bibsep}%
       \setlength{\parsep}{\z@}%
        \setlength{\partopsep}{0pt}%
        \setlength{\topsep}{0pt}}}
        {\end{enumerate}\vspace{-.6\baselineskip}}

\begin{document}
\pagenumbering{gobble} % suppress displaying page number

\name{陈超锋 | 个人简历}

% {E-mail}{mobilephone}{homepage}
% be careful of _ in emaill address
\contactInfo{cfchen@cs.hku.hk}{(+852) 92077137}{http://www.cfchen.com}
% {E-mail}{mobilephone}
% keep the last empty braces!
%\contactInfo{xxx@yuanbin.me}{(+86) 131-221-87xxx}{}
 
\section{\faGraduationCap\  教育背景}
\datedsubsection{\textbf{香港大学}~中国香港}{2015.09 -- 至今}
\textit{在读博士生}\ 计算机科学
\begin{itemize}
  \item 研究课题:唇语识别,人脸素描,场景文字识别
  \item 导师:Dr. Kenneth K.Y. Wong
\end{itemize}
\datedsubsection{\textbf{华中科技大学}~中国武汉}{2011.09 -- 2015.07}
\textit{计算机工程学士}\ 计算机科学
\begin{itemize}
  \item 毕业课题:视觉关注区域中的内容分析与图像场景分类
  \item 毕业导师:陈加忠教授
  \item 加权平均分:92.3,前1\%(300)
\end{itemize}

\section{\faHeart \ 研究兴趣}
\textbf{计算机视觉}:包括唇语识别,场景文字识别,人脸素描生成,图像风格迁移

\textbf{机器学习}:包括卷积神经网络,循环神经网络,对抗生成模型

\section{\faUsers\ 研究经历}
\datedsubsection{\textbf{香港大学}~中国香港}{2014.07 -- 2014.08}
\role{暑期实习生}{计算机系}
\begin{itemize}
  \item 研究项目:物体识别以及三维姿态估计
  \item 导师:Dr. Kenneth K.Y. Wong
\end{itemize}

\datedsubsection{\textbf{华中科技大学}~中国武汉}{2013.03 -- 2014.10}
\role{大学生创新项目}{计算机系}
\begin{itemize}
  \item 研究课题:基于双目立体匹配的深度图生成
  \item 导师:宋恩民教授
\end{itemize}

% Reference Test
%\datedsubsection{\textbf{Paper Title\cite{zaharia2012resilient}}}{May. 2015}
%An xxx optimized for xxx\cite{verma2015large}
%\begin{itemize}
%  \item main contribution
%\end{itemize}
\section{\faFileTextO\ 发表论文}

\begin{bibsection}
    \item W. Liu, \textbf{C. Chen}, K.-Y. K. Wong, Z. Su and J. Han. STAR-Net: A SpaTial Attention Residue Network for Scene Text Recognition. \emph{British Machine Vision Conference (BMVC)}, 2016.
\end{bibsection}
\vspace{.1in}

\section{\faTrophy\ 获奖情况}
\textbf{香港大学}~中国香港
\begin{itemize}
  \item 香港博士研究生奖学金 (HKPF) \hfill 2015 -- 2018
\end{itemize}

\vspace*{.1in}
\textbf{华中科技大学}~中国武汉
\begin{itemize}
  \item[\textbullet] 国家励志奖学金 \hfill 2012 -- 2013 | 2013 -- 2014
  \item[\textbullet] 启明学院特优生 \hfill 2012 -- 2013
  \item[\textbullet] 校三好学生 \hfill 2011 -- 2012 | 2012 -- 2013
  \item[\textbullet] 国家奖学金 \hfill 2011 -- 2012
\end{itemize}


\section{\faCogs\ 技能}
% increase linespacing [parsep=0.5ex]
\begin{itemize}[parsep=0.5ex]
  \item 编程: \texttt{Python, C/C++, Java, Matlab}
  \item 工具: PyTorch, Keras, Theano, LaTeX, OpenCV
  \item 语言: 普通话,英语
\end{itemize}

\section{\faInfo\ 社会活动}
\textbf{香港大学}~中国香港
\begin{itemize}
  \item 香港大学开放日学生助理 \hfill 2016.11 | 2017.11
  \item 香港创新嘉年华学生助理 \hfill 2016.11
\end{itemize}

\vspace*{.1in}
\textbf{华中科技大学}~中国武汉
\begin{itemize}
  \item 校医院志愿者 \hfill 2011 -- 2014
  \item 网络中心兼职网管 \hfill 2012 -- 2013
\end{itemize}
%% Reference
\end{document}
